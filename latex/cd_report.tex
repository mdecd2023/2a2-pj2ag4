\documentclass[14pt,a4paper]{report}  %紙張設定
\usepackage{xeCJK}%中文字體模組
\setCJKmainfont{標楷體} %設定中文字體
%\setCJKmainfont{MoeStandardKai.ttf}
\newfontfamily\sectionef{Times New Roman}%設定英文字體
%\newfontfamily\sectionef{Nimbus Roman}
\usepackage{enumerate}
\usepackage{amsmath,amssymb}%數學公式、符號
\usepackage{amsfonts} %數學簍空的英文字
\usepackage{graphicx, subfigure}%圖形
\usepackage{fontawesome5} %引用icon
\usepackage{type1cm} %調整字體絕對大小
\usepackage{textpos} %設定文字絕對位置
\usepackage[top=2.5truecm,bottom=2.5truecm,
left=3truecm,right=2.5truecm]{geometry}
\usepackage{titlesec} %目錄標題設定模組
\usepackage{titletoc} %目錄內容設定模組
\usepackage{textcomp} %表格設定模組
\usepackage{multirow} %合併行
%\usepackage{multicol} %合併欄
\usepackage{CJK} %中文模組
\usepackage{CJKnumb} %中文數字模組
\usepackage{wallpaper} %浮水印
\usepackage{listings} %引用程式碼
\usepackage{hyperref} %引用url連結
\usepackage{setspace}
\usepackage{lscape}%設定橫式
\lstset{language=Python, %設定語言
		basicstyle=\fontsize{10pt}{2pt}\selectfont, %設定程式內文字體大小
		frame=lines,	%設定程式框架為線
}
%\usepackage{subcaption}%副圖標
\graphicspath{{./../images/}} %圖片預設讀取路徑
\usepackage{indentfirst} %設定開頭縮排模組
\renewcommand{\figurename}{\Large 圖.} %更改圖片標題名稱
\renewcommand{\tablename}{\Large 表.}
\renewcommand{\lstlistingname}{\Large 程式.} %設定程式標示名稱
\hoffset=-5mm %調整左右邊界
\voffset=-8mm %調整上下邊界
\setlength{\parindent}{3em}%設定首行行距縮排
\usepackage{appendix} %附錄
\usepackage{diagbox}%引用表格
\usepackage{multirow}%表格置中
%\usepackage{number line}
%=------------------更改標題內容----------------------=%
\titleformat{\chapter}[hang]{\center\sectionef\fontsize{20pt}{1pt}\bfseries}{\LARGE 第\CJKnumber{\thechapter}章}{1em}{}[]
\titleformat{\section}[hang]{\sectionef\fontsize{18pt}{2.5pt}\bfseries}{{\thesection}}{0.5em}{}[]
\titleformat{\subsection}[hang]{\sectionef\fontsize{18pt}{2.5pt}\bfseries}{{\thesubsection}}{1em}{}[]
%=------------------更改目錄內容-----------------------=%
\titlecontents{chapter}[11mm]{}{\sectionef\fontsize{18pt}{2.5pt}\bfseries\makebox[3.5em][l]
{第\CJKnumber{\thecontentslabel}章}}{}{\titlerule*[0.7pc]{.}\contentspage}
\titlecontents{section}[18mm]{}{\sectionef\LARGE\makebox[1.5em][l]
{\thecontentslabel}}{}{\titlerule*[0.7pc]{.}\contentspage}
\titlecontents{subsection}[4em]{}{\sectionef\Large\makebox[2.5em][l]{{\thecontentslabel}}}{}{\titlerule*[0.7pc]{.}\contentspage}
%=----------------------章節間距----------------------=%
\titlespacing*{\chapter} {0pt}{0pt}{18pt}
\titlespacing*{\section} {0pt}{12pt}{6pt}
\titlespacing*{\subsection} {0pt}{6pt}{6pt}
%=----------------------標題-------------------------=%             
\begin{document} %文件
\sectionef %設定英文字體啟用
\vspace{12em}
\begin{titlepage}%開頭
\begin{center}   %標題  
\makebox[1.5\width][s] %[s] 代表 Stretch the interword space in text across the entire width
{\fontsize{24pt}{2.5pt}國立虎尾科技大學}\\[18pt]
\makebox[1.5\width][s]
{\fontsize{24pt}{2.5pt}機械設計工程系}\\[18pt]
\sectionef\fontsize{24pt}{1em}\selectfont\textbf
{
\vspace{0.5em}
cd2023 2a-pj1ag14分組報告}\\[18pt]
%設定文字盒子 [方框寬度的1.5倍寬][對其方式為文字平均分分布於方框中]\\距離下方18pt
\vspace{1em} %下移
\fontsize{30pt}{1pt}\selectfont\textbf{網際足球泡泡機器人場景設計}\\
\vspace{1em}
\sectionef\fontsize{30pt}{1em}\selectfont\textbf
{
\vspace{0.5em}
Web-based bubbleRob Football Scene Design}
 \vspace{2em}
%=---------------------參與人員-----------------------=%             
\end{center}
\begin{flushleft}
\begin{LARGE}

\hspace{32mm}\makebox[5cm][s]
{指導教授:\quad 嚴\quad 家\quad 銘\quad 老\quad 師}\\[6pt]
\hspace{32mm}\makebox[5cm][s]
{班\qquad 級:\quad 四\quad 設\quad 二\quad 甲}\\[6pt]
\hspace{32mm}\makebox[5cm][s]
{學\qquad 生:\quad 李\quad 承\quad 翰\quad(41023121)}
\\[6pt]
\hspace{32mm}\makebox[5cm][s]
{\hspace{36.5mm}林\quad 建\quad 維\quad(41023134)}\\[6pt]
%設定文字盒子[寬度為5cm][對其方式為文字平均分分布於方框中]空白距離{36.5mm}\空白1em
\end{LARGE}
\end{flushleft}
\vspace{6em}
\fontsize{18pt}{2pt}\selectfont\centerline{\makebox[\width][s]
{中華民國\hspace{3em} 
112 \quad 年\quad 3\quad 月}}
\end{titlepage}
\newpage


%=------------------------摘要-----------------------=%
\renewcommand{\baselinestretch}{1.5} %設定行距
\pagenumbering{roman} %設定頁數為羅馬數字
\clearpage  %設定頁數開始編譯
\sectionef
\addcontentsline{toc}{chapter}{摘~~~要} %將摘要加入目錄
\begin{center}
\LARGE\textbf{摘~~要}\\
\end{center}
\begin{flushleft}
\fontsize{14pt}{20pt}\sectionef\hspace{12pt}\quad 由於矩陣計算、自動求導技術、開源開發環境、多核GPU運算硬體等這四大發展趨勢,促使AI領域快速發展,藉由這樣的契機,將實體機電系統透過虛擬化訓練提高訓練效率,再將訓練完的模型應用到實體上。\\[12pt]

\fontsize{14pt}{20pt}\sectionef\hspace{12pt}\quad 此作業是讓泡泡機器人可以在不同電腦上進行控制,需要對應上對方電腦的IP位置,在cmd裡打上IPconfig就可以看到自己的IP位置,由於電導的ip位置對上即可在對應的電腦上運行程式。\\[12pt]

\end{flushleft}
\begin{center}
\fontsize{14pt}{20pt}\selectfont 關鍵字: 類神經網路、強化學習、\sectionef CoppeliaSim、OpenAI Gym
\end{center}
\newpage

\newpage
%=------------------------誌謝----------------------=%
\renewcommand{\baselinestretch}{1.5} %設定行距
\pagenumbering{roman} %設定頁數為羅馬數字
\clearpage  %設定頁數開始編譯
\sectionef
\centerline{\LARGE\textbf{致~~謝}}
\addcontentsline{toc}{chapter}{致~~~謝} %將摘要加入目錄
\begin{flushleft}
\fontsize{14pt}{20pt}\sectionef\hspace{12pt}\quad 在此鄭重感謝製作以及協助本分組報告完成的所有人員,首先向嚴家明老師致謝,感謝老師上課的教導,以及解決我們的疑問,也感謝學長的範本,雖在中間有許多不懂的地方,也謝謝別組的同學指正才讓我們可以如期完成作業,最後是由本分組組員完成本報告,特此感謝。
\end{flushleft}



\begin{figure}[hbt!]
\begin{center}
\includegraphics[width=15cm]{謝謝}
\end{center}
\end{figure}
\newpage
%=-------------------------內容----------------------=%
\chapter{前言}
\renewcommand{\baselinestretch}{10.0} %設定行距
\pagenumbering{arabic} %設定頁號阿拉伯數字
\setcounter{page}{1}  %設定頁數
\fontsize{14pt}{2.5pt}\sectionef
\section{作業內容}
協同產品設計實習的 Project 1 目的是讓學員可以從 \\
https://mde.tw/cd2023/content/BubbleRob.html 導引練習中, 了解 CoppeliaSim 套件中的諸多功能以及用法, 其中包括利用近接感測器偵測障礙物, 並透過 Lua script 控制 bubbleRob 雙輪車的移動. 為了讓各組學員了解在多人協同模式下, 開發機電資產品流程中必須面臨的許多議題(若要直接在瀏覽器中建立多方協同的場景, 可以透過 remote API(導引) 與 Visualization Stream 功能).自 W5 起將建立一個由 pj1 各組組長所組成的統整作業, 目標是利用兩台 BubbleRob 雙輪車在一足球場景中進行對戰, 其中在雙方球門設置感測器, 雙方各有一名 BubbleRob 負責運球, 在規定時間內, 每進一球後, 即透過程式重新從球場中線發球重啟賽局. 其中各組必須設法配置計分板顯示比賽剩餘時間與比分。\\

\begin{figure}[hbt!]
\begin{center}
\includegraphics[width=15cm]{車車}
\caption{\Large 泡泡機器人主體 }
\label{泡泡機器人主體}
\end{center}
\end{figure}


\section{遊戲規則}
遊戲規則如下:\\

Pong game 的遊戲規則簡單,透過bubbleRob將球打入對方球門即得一分,時間內其中一方分數最高即可獲勝,當有一方滿五分時遊戲結束。
\\
\\
\\
\\
   
\begin{figure}[hbt!]
\begin{center}
\includegraphics[width=15cm]{兩車車}
\caption{\Large 主體和場地}
\label{主體和場地 }
\end{center}
\end{figure}
\newpage



\begin{figure}[hbt!]
\chapter{製作過程}
\end{figure}
\section{建立機器人}
\begin{figure}[hbt!]
\begin{center}
\includegraphics[width=16cm]{車車}
\caption{\Large 泡泡機器人}\label{泡泡機器人}
\end{center}
\end{figure}
\fontsize{14pt}{2.5pt}\sectionef
泡泡機器人的製作過程,首先先新增球體當作機器人主體,接下來新增兩軸當作馬達,然後分別裝上車輪,再來新增感測器,這時的機器人後面沒有支撐,所以我們在後面新增滑塊,車體就能保持平衡,這就是我們的泡泡機器人,詳細wink可以到\\
https://mdecd2023.github.io/2a-pj1ag5/content/tutorial1.html \\
\newpage
\section{建立球框}
\begin{figure}[hbt!]
\begin{center}
\includegraphics[width=16cm]{球框}
\caption{\Large 球框}\label{球框}
\end{center}
\end{figure}
球門我們利用NX來製作,這裡就不詳細說明,可以到\\
https://mdecd2023.github.io/2a-pj1ag5/content/project1.html\\
了解詳細過程 \\
\newpage
\section{程式}
\begin{figure}[hbt!]
\begin{center}
\includegraphics[width=16cm]{程式}
\caption{\Large 程式}\label{程式}
\end{center}
\end{figure}
關於程式的講解,這裡就不詳細說明,可以到\\
https://mdecd2023.github.io/2a-pj1ag5/content/project1.html\\
了解詳細過程 \\
\newpage

\begin{figure}[hbt!]
\chapter{心得感想}
\end{figure}
\section{李承翰41023121}
 這學期的課程節奏十分緊湊,我也花了許多的時間在了解程式及原理,雖然現在有AI的協助但是AI必須問得非常精準才能得到我想要的答案,像是程式出錯時的回應他能很精準地回答,但是如果是問他程式的問題時他有時會列出很多種可能,看到其實心很累啊。這次的車車老實說大部分還是由老師所提供也謝謝同學的教導與修正我才能如期完成這項作業。\\[6pt]

\section{林建維41023134}
 這次的教學比較難,花了很多時間來處理解決,也謝謝同學的凱瑞才得以完成。\\
\begin{figure}[hbt!]
\begin{center}
\includegraphics[height=6cm]{h}
\end{center}
\end{figure} 


%=----------------書背----------------------=%
%\newpage
%\begin{landscape} %橫式環境
%\begin{center}
%\fontsize{0.001pt}{1pt}\selectfont .
%\vspace{70mm}
%\rotatebox[origin=cc]{90}{\LARGE 【14】}\rotatebox[origin=cc]%{180}{\LARGE 1-2-APP-8765} %旋轉
%\end{center}
%\end{landscape}
\end{document}

