\documentclass[10pt]{article}
\usepackage[utf8]{inputenc}
\usepackage[T1]{fontenc}
\usepackage{CJKutf8}
\usepackage{multirow}
\usepackage{amsmath}
\usepackage{amsfonts}
\usepackage{amssymb}
\usepackage[version=4]{mhchem}
\usepackage{stmaryrd}

\title{國立虎尾 科技 大 學 }


\author{\begin{CJK}{UTF8}{mj}分組報告製作修習學生\end{CJK}: \begin{CJK}{UTF8}{mj}四設二甲\end{CJK} 41023191 \begin{CJK}{UTF8}{mj}第一位\end{CJK}\\
\begin{CJK}{UTF8}{mj}四設二甲\end{CJK} 41023192 \begin{CJK}{UTF8}{mj}第二位\end{CJK}\\
\begin{CJK}{UTF8}{mj}四設二甲\end{CJK} 41023193 \begin{CJK}{UTF8}{mj}第三位\end{CJK}\\
\begin{CJK}{UTF8}{mj}四設二甲\end{CJK} 41023194 \begin{CJK}{UTF8}{mj}第四位\end{CJK}\\
\begin{CJK}{UTF8}{mj}四設二甲\end{CJK} 41023195 \begin{CJK}{UTF8}{mj}第五位\end{CJK}}
\date{}


\begin{document}
\maketitle
\section{\begin{CJK}{UTF8}{mj}機械設計工\end{CJK} \begin{CJK}{UTF8}{mj}程\end{CJK} \begin{CJK}{UTF8}{mj}系\end{CJK} cd2023 2a-pj1ag1 \begin{CJK}{UTF8}{mj}分組報告\end{CJK}}
\section{\begin{CJK}{UTF8}{mj}網際手足球場景設計\end{CJK}}
\section{Web-based Foosball Scene Design}
\begin{center}
\begin{tabular}{|c|c|c|c|c|c|c|}
\hline
\multicolumn{2}{|c|}{\begin{CJK}{UTF8}{mj}指渞教授\end{CJK}} & \multirow{2}{*}{\begin{CJK}{UTF8}{mj}嚴\end{CJK}} & \multirow{2}{*}{$\begin{array}{l}\text { 家 } \\
\text { 設 }\end{array}$} & \multirow{2}{*}{$\begin{array}{l}\text { 銘 } \\
\text { 二 }\end{array}$} & \multicolumn{2}{|c|}{\multirow{2}{*}}{$\begin{array}{ll}\text { 老 } & \text { 師 } \\
\text { 甲 }\end{array}$} \\
\hline
\begin{CJK}{UTF8}{mj}班\end{CJK} & \begin{CJK}{UTF8}{mj}級\end{CJK} &  &  &  &  \\
\hline
\multirow[t]{5}{*}{T} & \begin{CJK}{UTF8}{mj}生\end{CJK} & \begin{CJK}{UTF8}{mj}第\end{CJK} & - & \begin{CJK}{UTF8}{mj}位\end{CJK} & (41 & $023191)$ \\
\hline
 &  & \begin{CJK}{UTF8}{mj}第\end{CJK} & \begin{CJK}{UTF8}{mj}二\end{CJK} & \begin{CJK}{UTF8}{mj}位\end{CJK} & (41 & 023192) \\
\hline
 &  & \begin{CJK}{UTF8}{mj}第\end{CJK} & \begin{CJK}{UTF8}{mj}三\end{CJK} & \begin{CJK}{UTF8}{mj}位\end{CJK} & (41) & $023193)$ \\
\hline
 &  & o & \begin{CJK}{UTF8}{mj}四\end{CJK} & \begin{CJK}{UTF8}{mj}位\end{CJK} & $(4)$ & 023194) \\
\hline
 &  & \begin{CJK}{UTF8}{mj}第\end{CJK} & \begin{CJK}{UTF8}{mj}五\end{CJK} & \begin{CJK}{UTF8}{mj}伩\end{CJK} & $(4)$ & $023195)$ \\
\hline
\end{tabular}
\end{center}

\begin{CJK}{UTF8}{mj}中華民國\end{CJK}

112 \begin{CJK}{UTF8}{mj}年\end{CJK} 3 \begin{CJK}{UTF8}{mj}月\end{CJK} \begin{CJK}{UTF8}{mj}國立虎尾科技大學\end{CJK} \begin{CJK}{UTF8}{mj}機械設計工程系\end{CJK} \begin{CJK}{UTF8}{mj}分組報告製作合格認可證明\end{CJK}

\begin{CJK}{UTF8}{mj}分組報告題目\end{CJK}:\begin{CJK}{UTF8}{mj}網際手足球場景設計\end{CJK}

\begin{CJK}{UTF8}{mj}經評量合格\end{CJK},\begin{CJK}{UTF8}{mj}特此證明\end{CJK}

\begin{CJK}{UTF8}{mj}評\end{CJK} \begin{CJK}{UTF8}{mj}審\end{CJK} \begin{CJK}{UTF8}{mj}委\end{CJK} \begin{CJK}{UTF8}{mj}員\end{CJK} :

\begin{CJK}{UTF8}{mj}指導\end{CJK} \begin{CJK}{UTF8}{mj}老\end{CJK} \begin{CJK}{UTF8}{mj}師\end{CJK}:

\begin{CJK}{UTF8}{mj}中華民國一一二年三月\end{CJK} \begin{CJK}{UTF8}{mj}三十一日\end{CJK} \begin{CJK}{UTF8}{mj}由於矩陣計算\end{CJK}、\begin{CJK}{UTF8}{mj}自動求導技術\end{CJK}、\begin{CJK}{UTF8}{mj}開源開發環境\end{CJK}、\begin{CJK}{UTF8}{mj}多核\end{CJK}GPU\begin{CJK}{UTF8}{mj}運算硬體\end{CJK} \begin{CJK}{UTF8}{mj}等這四大發展趨勢\end{CJK}, \begin{CJK}{UTF8}{mj}促使\end{CJK} $\mathrm{AI}$ \begin{CJK}{UTF8}{mj}領域快速發展\end{CJK},\begin{CJK}{UTF8}{mj}藉由這樣的契機\end{CJK},\begin{CJK}{UTF8}{mj}將實體\end{CJK} \begin{CJK}{UTF8}{mj}機電系統透過虛擬化訓練提高訓練效率\end{CJK},\begin{CJK}{UTF8}{mj}再將訓練完的模型應用到實體\end{CJK} \begin{CJK}{UTF8}{mj}上\end{CJK}。

\begin{CJK}{UTF8}{mj}此專題是運用實體冰球對打機\end{CJK}, \begin{CJK}{UTF8}{mj}將其導入\end{CJK} CoppeliaSim \begin{CJK}{UTF8}{mj}模擬環境並給\end{CJK} \begin{CJK}{UTF8}{mj}予對應設置\end{CJK},\begin{CJK}{UTF8}{mj}將其機電系統简化並運用\end{CJK} Open AI Gym \begin{CJK}{UTF8}{mj}進行訓練\end{CJK},\begin{CJK}{UTF8}{mj}找到適\end{CJK} \begin{CJK}{UTF8}{mj}合此系統的演算法後\end{CJK}, \begin{CJK}{UTF8}{mj}再到\end{CJK} CoppeliaSim\begin{CJK}{UTF8}{mj}模擬環境中進行測試演算法在\end{CJK} \begin{CJK}{UTF8}{mj}實際運用上的可行性\end{CJK}。\begin{CJK}{UTF8}{mj}並嘗試透過架設伺服器將\end{CJK} CoppeliaSim \begin{CJK}{UTF8}{mj}影像串流\end{CJK} \begin{CJK}{UTF8}{mj}到網頁供使用者觀看或操控\end{CJK}。

\begin{CJK}{UTF8}{mj}關鍵字\end{CJK}:\begin{CJK}{UTF8}{mj}類神經網路\end{CJK}、\begin{CJK}{UTF8}{mj}強化學習\end{CJK}、CoppeliaSim、OpenAI Gym

\begin{abstract}
Due to the four major development trends of multidimensional arrays computing, automatic differentiation, open source development environment, and multi-core GPUs computing hardware. The rapid development of the AI field has been promoted. In view of this development, the physical mechatronic systems can gain machine learning efficiency through their simulated virtual system training process. And afterwards to apply the trained model into real mechatronic systems.
This project is to use the physical air hockey to play machine, introduce it into the CoppeliaSim simulation environment and give the corresponding settings, simplify its electromechanical system and use Open AI Gym for training, find an algorithm suitable for this system, and then perform it in the CoppeliaSim simulation environment Feasibility of testing algorithm in practical application. And try to stream CoppeliaSim images to web pages for users to watch or manipulate by setting up a server.
\end{abstract}

Keyword: nerual network reinforcement learning C CoppeliaSim OpenAI Gym

\section{\begin{CJK}{UTF8}{mj}誌\end{CJK} \begin{CJK}{UTF8}{mj}謝\end{CJK}}
\begin{CJK}{UTF8}{mj}在此鄭重感謝製作以及協助本分組報告完成的所有人員\end{CJK},\begin{CJK}{UTF8}{mj}首先向大\end{CJK} \begin{CJK}{UTF8}{mj}三學長致謝\end{CJK}, \begin{CJK}{UTF8}{mj}他們不辭辛勞解決我們的提問\end{CJK}, \begin{CJK}{UTF8}{mj}甚至從來沒有不耐煩\end{CJK}, \begin{CJK}{UTF8}{mj}總\end{CJK} \begin{CJK}{UTF8}{mj}是貼心為我們找出最佳解答\end{CJK}。\begin{CJK}{UTF8}{mj}再來是我們的分組組長\end{CJK}, \begin{CJK}{UTF8}{mj}他給了我們全方\end{CJK} \begin{CJK}{UTF8}{mj}位的支援\end{CJK}, \begin{CJK}{UTF8}{mj}提供我們解決問題的方向和建議\end{CJK}, \begin{CJK}{UTF8}{mj}給予開始接觸網際對戰遊\end{CJK} \begin{CJK}{UTF8}{mj}戲的我們有個學習的方向\end{CJK}, \begin{CJK}{UTF8}{mj}開會時也時不時向我們提出建議以及末來走\end{CJK} \begin{CJK}{UTF8}{mj}向\end{CJK},\begin{CJK}{UTF8}{mj}同時也給了我們能自由摸索的空間及時間\end{CJK}, \begin{CJK}{UTF8}{mj}最後是由本分組組員同\end{CJK} \begin{CJK}{UTF8}{mj}心協力才得以完成本報告\end{CJK},\begin{CJK}{UTF8}{mj}特此感謝\end{CJK}。


\end{document}